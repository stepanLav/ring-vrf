\subsection{Ring VRF in the UC Model}


In this section, we describe the security of our new cryptographic primitive ring VRF. We describe the  ring VRF in the real world and in the ideal world. 

\paragraph{Ring VRF in the real world:}
\begin{definition}[Ring-VRF (rVRF)]\label{def:ringvrf}
	%TODO ADD anonymous key here
	A ring VRF is a VRF with a  function $ F(.):\{0,1\}^* \rightarrow \{0,1\}^{\ell_\rvrf} $ and with the following algorithms:
	
	\begin{itemize}
		\item $ \rvrf.\keygen(1^\secpar) \rightarrow (\sk,\pk)$ where $ \secpar $ is the security parameter,
	\end{itemize}
	Given list of public keys $ \comring = \set{\pk_1, \pk_2, \ldots, \pk_n}$, a message $ m \in \{0,1\}^{\ell_m} $
	\begin{itemize}
		\item $ \rvrf.\sign(\sk_i, \comring, m)\rightarrow (\sigma,W) $ where  $\sigma $ is a signature of the message $ m $ signed by $ \sk_i, \comring $ and $ W $ is an anonymous key.
		\item $ \rvrf.\verify(\comring,W, m,\sigma) \rightarrow  (b, y)$ where $ b \in \{0,1\} $ and $ y \in \{0,1\}^{\ell_\rvrf}\cup \{\perp\} $. $ b =1 $ means verified and $ b = 0 $ means not verified.
		\item $ \rvrf.\link(\sk_i, \comring,W,m, \sigma) \rightarrow \hat\sigma $ where  $ \hat\sigma $ is a signature that links signer of the ring signature $ \sigma $. 
		\item $ \rvrf.\link\verify( \pk_i,\comring,W, m, \sigma, \hat\sigma)\rightarrow b$ where $ b \in \{0,1\} $. $ b =1 $ means verified and $ b = 0 $ means not verified.
	\end{itemize}
\end{definition}

\paragraph{Ring VRF in the ideal world:} 
%A ring VRF operates like a VRF but only proves its key comes from a specific list without giving any information about which specific key. 
We define the ring VRF functionality $ \fgvrf $ in Figure \ref{f:gvrf}. The functionality lets parties generate a key (Key Generation), evaluate a message with the party's key (Ring VRF Evaluation), sign a message by one of the keys (Ring VRF signature) and verify the signature and obtain the evaluation output without knowing the key used for the signature and evaluation (Ring VRF Verification). We also define linking procedures in $ \fgvrf $ to link a signature with its associated key. So, if a party wants to reveal its identity at some point, it can use the linking process to show that the evaluation is executed with its key (Linking Signature). Later on, anyone can verify the linking signature (Linking Verification).

In a nutshell, the functionality $\fgvrf$, when given as input a message $m$ and a key set $\comring$ (that we call a ring) of party, allows to create $n$ possible different outputs pseudo-random that appear independent from the inputs. The output can be verified to have been computed correctly by one of the participants in $\comring$ without revealing who they are. At a later stage, the author of the ring VRF output can prove that the output was generated by him and no other participant could have done so.
\begin{figure}\scriptsize
	\begin{tcolorbox}
		{  $ \fgvrf $ runs two PPT algorithms $ \gen_W$ and $\gen_{sign} $ during the execution.
			
			 \begin{description}
				
				\item[Key Generation.] upon receiving a message $(\msg{keygen}, \sid)$ from a party $\user_i$, send $(\msg{keygen}, \sid, \user_i)$ to the simulator $\simulator$.
				Upon receiving a message $(\msg{verificationkey}, \sid, \pk)$ from $\simulator$, verify that $\pk$ has not been recorded before; then, store in the table $\vklist$, under $\user_i$, the value $\pk$.
				Return $(\msg{verificationkey}, \sid, \pk)$ to $ \user_i$.
				
				\item[Malicious Key Generation.] upon receiving a message $(\msg{keygen}, \sid, \pk)$ from $\simulator$, verify that $\pk$ was not yet recorded, and if so record in the table $\vklist$ the value $\pk$ under $\simulator$. Else, ignore the message.
				
				%\item[Honest Ring VRF Evaluation.] upon receiving a message $(\msg{eval}, \sid, \comring, \pk_i, m)$ from $\user_i$, verify that 
				%$\pk_i \in \comring$ 
				%and  
				%there exists $ \pk_i $ in $\vklist $ associated with $ \user_i $. If that was not the case, just ignore the request.
				%If there exists no $ W $ such that $ \anonymouskeymap[W] = (m, \comring, \pk_i) $, let $ W \sample \bin^\secpar $ and  $y \sample \bin^{\ell_\rvrf}$. Then, set $ \evaluationslist[m, W] = y$ and $ \anonymouskeymap[W] = (m, \comring,\pk_i) $.
				%Return $(\msg{evaluated}, \sid, \comring, m, W, y)$ to $ \user_i $.
				%The functionality does not check whether the evaluater's public key is in the ring because here we consider m, \comring as an input of the evaluation which is evaluated by a party who is not neccesarily in the ring. 
				\item[Malicious Ring VRF Evaluation.] upon receiving a message $(\msg{eval}, \sid, \comring, \pk_i, W, m)$ from $\sim$, verify that $ \pk_i $ has not been recorded under an honest party's key.
			    If it is the case record in the table $\vklist$ the value $\pk_i$ under $\simulator$. Else, ignore the request.  If $ \counter[m,\comring] $ does not exist, initiate $ \counter[m,\comring] = 0 $.
			    If there exists $ W $ such that $ \anonymouskeymap[W] = (m',\comring', \pk)$ then do the following:
			    \begin{itemize}
			    	\item if $(m', \comring', \pk) \neq  (m, \comring, \pk_i)  $, ignore the request,
			    	\item else obtain $ y = \evaluationslist[m, W]   $. 
			    \end{itemize}
				If there exists no $ W $ such that $ \anonymouskeymap[W] = (m, \comring, \pk_i) $, let   $y \sample \bin^{\ell_\rvrf}$ and increment $ \counter[m,\comring] $. Then, set $ \evaluationslist[m, W] = y$, $ \anonymouskeymap[W] = (m, \comring,\pk_i) $ .
				Return $(\msg{evaluated}, \sid, \comring, m, W, y)$ to $ \user_i $.
				
				
				
				
				\item[Honest Ring VRF Signature.] upon receiving a message $(\msg{sign}, \sid, \comring, \pk_i, m)$ from $\user_i$, verify that $\pk_i \in \comring$ and that there exists a public key $\pk_i$ associated to $\user_i$ in the table $ \vklist $. If that wasn't the case, just ignore the request. 	
				If there exists no $ W' $ such that $ \anonymouskeymap[W'] = (m, \comring, \pk_i) $, run $ \gen_W(\comring, \pk_i, m) \rightarrow W$. Then, let $y \sample \bin^{\ell_\rvrf}$ and set $ \anonymouskeymap[W] = (m, \comring,\pk_i) $ and set $ \evaluationslist[m, W] = y$.
%					\begin{itemize}
%						%\item If there exists $ W \in  \anonymouskeymap  $, abort.
%						\item Else 
%						%TODO define what \in \anonymouskeymap mean
%					\end{itemize}
%			    \end{itemize}
				Obtain $ W, y $ where  $ \evaluationslist[m, W] = y$, $ \anonymouskeymap[W] = (m, \comring,\pk_i) $ and run  $ \gen_{sign}(\comring, W, m) \rightarrow \sigma $. Verify that $ [m, W, \sigma, 0] $ is not recorded. If not, abort. Otherwise, record $ [m, W, \sigma, 1] $. Return $(\msg{signature}, \sid, \comring,W,m, y, \sigma)$ to $\user_i$.
				
				%\item[Malicious VRF evaluation.] upon receiving a message $(\msg{evalprove}, \sid, \comring, m)$ from $\simulator$, check that $\vklist$ has a public key associated to $\simulator$. If not, ignore the request. If $\evaluationslist[\comring, m][\simulator]$ is not set, sample $y \sample \bin^{\ell(\secpar)}$ and set $\evaluationslist[\comring, m][\simulator] \defeq y$ (and $\signaturelist[\comring,m]$ to $\emptyset$). If $\signaturelist[\comring, m]$ contains a proof (i.e., if $\signaturelist[\comring, m]$ is not empty), return $(\msg{evaluated}, \sid, y)$ to $\simulator$. Else, ignore the request.
				
				%\item[Verification.] upon receiving a message $(\msg{verify}, \sid, \comring, m, y, \sigma)$, from any party forward the message to the simulator. If there exists a $\pk_i$ among the values of \texttt{verification\_keys}, and there exists $\sigma \in \signaturelist[\comring, m]$, set $b = 1$. Else, set $b =0$. Finally, output $(\msg{verified}, \sid, \comring, m, y, \sigma, b)$.
				\item[Ring VRF Verification.] upon receiving a message $(\msg{verify}, \sid, \comring,W, m, \sigma)$ from a party, relay the message $(\msg{verify}, \sid, \comring,W, m, \sigma)$ to $ \simulator $ and receive back the message $(\msg{verified}, \sid, \comring,W, m, \sigma, b_{\simulator}, \pk_\simulator)$. Then do the following: 
				\begin{enumerate}[label={{Cond.-} }{{\arabic*}}, start = 1]
					\item If there exits a record $ [m,W,\sigma, 1] $, set $ b = 1 $. (This condition guarantees the completeness meaning that if  $ W $ is an anonymous key that is generated for the ring $ \comring $ and  the message $ m $ and the signature $ \sigma $ is legitimately generated for $ m, W $, then the verification succeeds.)
					\item Else if $ \pk  $ is an honest verification key where $ \anonymouskeymap[W] = (.,., \pk) $ and there exists no record $ [m, \comring, W, \sigma', 1] $ for any $ \sigma' $, then let $ b= 0  $.
					(This condition guarantees unforgeability meaning that if an honest party never signs a message $ m $ for a ring $ \comring $, then the verification fails.)\label{cond:forgery}
					\item Else if there exists a record  such as $ [m,W,\sigma, b'] $, set $ b = b' $. (This condition guarantees consistency meaning that all identical verification requests will output the same $ b $) \label{cond:consistency}
					\item Else if $ \pk  $ is an honest verification key where $ \anonymouskeymap[W] = (.,., \pk) $ and there exists a record $ [m, W, \sigma', 1] $ for any $ \sigma' $, then let $ b= b_{\simulator} $ and record $ [m, W,\sigma, b_{\simulator}] $. (This condition guarantees that if $ m $ is signed by an honest party for the ring $ \comring $ at some point and the signature is $ \sigma' \neq \sigma $, then the decision of verification is up to the adversary) \label{cond:differentsignature}
					\item \label{cond:forgerymalicious}Else if there exists $ \anonymouskeymap[W] = (m', \comring',.)  $ where $ (m', \comring') \neq (m, \comring) $ or $ \counter[m, \comring] > |\comring_m| $ where $ \comring_m $ is a set of keys in $ \comring $ which are not honest or $ b_{\simulator} = 0 $ or $ \pk_\simulator $ belongs to an honest party, set $ b = 0 $ and record $ [m, \comring,W,\sigma, 0] $. (This condition guarantees that if $ W $ is an anonymous key of a different message and ring or the number of anonymous keys of malicious parties in $ \comring $ is more than their number or     $ \simulator $ does not verify $ \sigma $, then the verification fails.)
					\item Else set $ b = 1 $. Set $ \evaluationslist[m, W]\sample \bin^{\ell_\rvrf}$, $ \anonymouskeymap[W]  = (m, \comring, \pk_\simulator)$ and $ \counter[m, \comring, \pk_\simulator] = 0 $ if they are not defined before. Increment $ \counter[m, \comring, \pk_\simulator]  $. \label{cond:advsignature}
				\end{enumerate}
				In the end, if $ b = 0 $, set $ \out = \emptyset $. Otherwise,  set $ \out = \evaluationslist[m, W]$. 		Finally, output $(\msg{verified}, \sid, \comring,W, m, \sigma, \out, b)$ to the party.
				
			\end{description}
		
			
		}
	\end{tcolorbox}
	\caption{Functionality $\fgvrf$.\label{f:gvrf}}
\end{figure}
	


\begin{figure}\scriptsize
	\begin{tcolorbox}
		{  This part of $ \fgvrf $ for the parties who want to show that they generate a particular ring signature.
			
		
			\begin{description}
				\item[Linking signature.] upon receiving a message $(\msg{link}, \sid, \comring, \pk_i, W, m,\sigma)$ from $\user_i$, check that $\pk_i $ is associated to $\user_i$ in $ \vklist $, $ \anonymouskeymap[W] = (m, \comring, \pk_i) $ and 
				check whether $ [m, W, \sigma, 1] $ is stored. If any of them fails, ignore the request. Otherwise,
				send $(\msg{link}, \sid, \comring, W, m, y)$ to $\simulator$. Upon receiving $(\msg{linkproof}, \sid, \comring, W, m, y, \hat \sigma)$ from $\simulator$, verify that $ [m, \comring, \pk_i, \sigma, \hat{\sigma}, 0] $ is not stored in $ \Linklist $. If not, abort. Otherwise,  record $\hat\sigma$ to $[m, \comring, \pk_i,\sigma, \hat{\sigma}, 1]$ to $ \Linklist $ and return $(\msg{linked}, \sid, \comring, \pk_i,W, m, y,\sigma, \hat\sigma)$ to $\user_i$.
				%\item[Malicious linking proof.] upon receiving a message $(\msg{link}, \sid, \comring, m, y)$ from $\simulator$, check that $\vklist$ has a key set for $\simulator$, and that it is in $R$.
				%Check that $\evaluationslist[\comring, m][\simulator] = y$.
				%If any of the above is not satisfied, ignore the request.
				%Return $(\msg{linked}, \sid, y)$ to $\simulator$.
				\item[Linking verification.] upon receiving a message $(\msg{verifylink}, \sid, \pk_i, \comring, W, m,\sigma,\hat\sigma)$ from any party forward the message to the simulator and receive back  the message $(\msg{verified}, \sid, \pk_i, \comring, W,m, \sigma,\hat\sigma,  b_{\simulator})$. Then do the following:
				
				\begin{itemize}
					\item If there exits a record $ [m, \comring,\pk_i,\sigma,\hat\sigma, 1] $ in $ \Linklist $, set $ b = 1 $ and $ \pk = \pk $. (This condition guarantees the completeness.)
					\item Else if $ \pkvrf_i $ is a key of an honest party and there exits no record such as $ [m, \comring,\pk_i,\sigma,\hat\sigma',  1] $ for any  $  \hat\sigma'$, then set $ b = 0 $ and record $ [m, \comring,\pk_i,\sigma,\hat\sigma,  0] $. (This condition guarantees unforgeability meaning that if an honest party never signs a message $ m $ in the linking signature, then the verification fails.)
					\item Else if there exists a record  such as $ [m, \comring,\pk_i,\sigma,\hat\sigma,  b'] $, set $ b = b' $. 
					\item Else set $ b = b_{\simulator} $ and record $ [m, \comring,\pk_i,\sigma,\hat\sigma,  1] $. 
				\end{itemize}
				
				Return $(\msg{verified}, \sid, \pk_i, \comring, m, \hat\sigma, b).$ to the party.
			\end{description}
		}
	\end{tcolorbox}
	\caption{Functionality $\fgvrf$.\label{f:gvrf}}
\end{figure}



We give some remarks related to our functionality:

\begin{enumerate}[label={{R-} }{{\arabic*}}, start = 1]
	
	%\item The ring VRF signature does not need to be random but it must be \emph{unique}  for its ring and the message. The reason of it to have a mapping from a ring VRF signature to its evaluation output. The map is necessary for $ \fgvrf $ to output the corresponding evaluation value for the signature during the verification process i.e, $ [m, \comring, \sigma] \rightarrow \pk, \evaluationsecretlist[m, \comring][\pkvrf] \rightarrow y $.
	\item In classical VRF, a VRF $ F $ is a deterministic function which maps a message and a public key to a random output. While in ring VRF, a message, a public key and a ring map to a random value, the verification algorithm of a ring VRF does not take the key as an input because it should be hidden. Therefore, the verification should be executed without the public key.  So, the functionality $ \fgvrf $ needs to find a way to verify the ring VRF output of a message, a public key and a ring map without knowing the public key. Because of this, $ \fgvrf $ generates an anonymized key $ W $ for each evaluation so that a message $ m $  and $ W $ maps to the random output. One can imagine this  as if a VRF output is generated with the input message $ m $ and the key $ W $ as in classical VRF i.e.,  $ F(m, W) $. 
	
	\item  If an honest party signs a message for a ring and obtains a signature, $ \fgvrf $ allows the simulator to generate another signature in \ref{cond:differentsignature} if the simulator wants. We remark that this is not a security issue because an honest party has already committed to sign the message.  A similar condition  exists in the EUF-CMA secure signature functionality $ \fsig $ \cite{canettiFsig}.
	
	\item \ref{cond:advsignature} of the ring VRF verification process covers the case where the adversary decides whether accepting the signature generated for its key if  it could be a valid signature for the ring i.e., the malicious key is in the ring and the anonymous key in the verification request is unique.
	
	\item The linking signature and the linking verification works similar to the EUF-CMA secure signature functionality $ \fsig $ \cite{canettiFsig}.
	
	
\end{enumerate}


\begin{definition}[Anonymous $ \fgvrf $]\label{def:anonymity}
	We call that $ \fgvrf $ is anonymous if the outputs of $ \gen_{sign} $ and $ \gen_W $ are pseudo-random.
	%TODO define this more formally.
\end{definition}




	

