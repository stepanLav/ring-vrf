\newcommand{\com}{\mathsf{Com}}
\newcommand{\open}{\mathsf{Open}}
\newcommand{\commit}{\mathsf{Commit}}
\section{Preliminaries} %It is better name because this section contains information that helps reader to understand the later sections better
\label{sec:background}
We   give definitions of some cryptographic primitives that help us to construct our ring VRF protocol.

% underlying both Chaum-Pedersen DLEQ proofs and pairing based zkSNARKs.

% As our ring VRF is built by composing them,
% We briefly recall the primitives and security assumptions underlying
% both Chaum-Pedersen DLEQ proofs and pairing based zkSNARKs.

\begin{definition}[Commitment Scheme]
	A commitment scheme $ \com $ consists of two algorithms:
	\begin{itemize}
		\item $ \com.\commit(x,r) \mapsto c$ creates a commitment to $ x $ with a randomness $ r $.
		\item $ \com.\open(c;x,r)  \mapsto x'$ opens the commitment $ c $ with the openings $ x,r$ to $ x' $. If $ x = x' $, then $ x,r $ are the correct openings.
	\end{itemize}
\end{definition}
%\subsection{Elliptic curves}
%\label{sec:ec_background}

% As our ring VRF is built by composing them, 
% We briefly recall the primitives and security assumptions underlying
% both Chaum-Pedersen DLEQ proofs and pairing based zkSNARKs. 

%We obey mathematical and cryptographic implementation convention by 
%adopting additive notation for elliptic curve and multiplicative notation
%for elliptic curve scalar multiplications and pairing target groups.
%
%All object implicitly depend a security parameter \secparam.
%All protocols therefore have an implicit parameter generation algorithm,
%which output their hash functions, elliptic curves, and
% some independent base points on the elliptic curves.
%
%We need an elliptic curve $\ecE$ over a field of characteristic $q$,
%equipped with a type III pairing $e : \grE_1 \times \grE_2 \to \grE_T$,
%where the groups  $\grE_1 \le \ecE[\F_q]$, $\grE_2 \le \ecE[\F_{q^2}]$, and
%$\grE_T \le \F^*_{q^{12}}$ all have prime order $p \approx 2^{2\secparam}$.
%
%We write $\grE$ when discussing the Chaum-Pedersen DLEQ proofs, which do
%not employ pairings, but $\grE$ always denotes $\grE_1$ eventually.
%We avoid pairing unfriendly assumptions like DDH of course, but really
%we employ the algebraic group model (AGM) throughout.
%
%We sweep cofactor concerns under the rug when discussing Groth16,
%where our pairings demand deserialization prove group membership in $\grE_1$
%or $\grE_2$.  We explicitly multiply by the effective cofactor $h$ when
%doing Chaum-Pedersen DLEQ proofs though, as not doing so risks miss-reading
%by implementers.  Yet, this becomes redundant if deserialization proves
%group membership, meaning $h=1$.
%
%We also let $\ecJ$ denote a ZCash Sapling style ``JubJub'' Edwards curve
%over $\F_p$, with distinguished subgroup $\grJ$ of prime order $p_\grJ$, so that
%SNARKs on $\ecE$ prove $\grJ$ arithmetic relatively cheaply.
%Aside from Jubjub, we optionally want a ``sister'' Edwards curve $\ecEsis$,
%with a subgroup $\grE$ of the same order $p$ as $\grE_1$,
% but which lacks any pairing.
%
%We let $H_p : \{0,1\}^* \to \F_p$ and $H_{\grEsis} : \{0,1\}^* \to \grEsis$
%denote a hash-to-scalar and a hash-to-curve  with ranges
%$\F_p$ and $\grEsis$, respectively, always modeled as random oracles.  
%We only ever hash-to-$\ecEsis$ because hash-to-$\grE_1$ create a miss-use
%footgun for an anonymity protocol.  Also hash-to-$\ecEsis$ is faster.
%We let $H'$ denote the hash to the VRF output space, usually
% a key derivation function plus a stream cipher, also modeled as a random oracle.
%
%All our security proofs ignore these underlying elliptic curve concerns,
%so $\grE_1 = \grEsis$ and cofactors are ignores.
%All hashes are random oracles.  DDH is hard in $\grE_1$ and $\grJ$.
%AGM is used for $\ecE$ in Groth16 sections, or wherever is convenient. 




%\subsection{Non-Interactive Zero-Knowledge Proofs}
%\label{subsec:zkp_background}

% refs.
% https://people.csail.mit.edu/silvio/Selected%20Scientific%20Papers/Zero%20Knowledge/Noninteractive_Zero-Knowkedge.pdf
%   Alright but kinda poorly phrases
% https://inst.eecs.berkeley.edu/~cs276/fa20/notes/Multiple%20NIZK%20from%20general%20assumptions.pdf
%   Addresses the ZK definitions better
% 

We let $ (\rel,z) $ denote a output of a relation generator $ \mathfrak{R} $ which receives $ \secparam $ as an input. $ \rel $ is a polynomial time decidable relation and $ z $ is an auxiliary input. For $ (x;\omega) \in \rel $, we call that $ x$ is the statement and $ \omega $ is the witness.
 %All non-interactive zero-knowledge proof systems have some setup procedure
%$\mathtt{Setup}$ that takes some implicit parameters and some ``circuit''
%description of \rel, and may produces a structured reference string (SRS).
%We discuss SRSes and their toxic waste in \S\ref{sec:rvrf_cont} but
%SRSes remain implicit in our notation.
A non-interactive proof system $ \NIZK $ for $\rel$ consists of the following algorithms:
\begin{itemize}
\item $\NIZK.\Setup(\rel) \rightarrow (crs, \tau)$  given $ \rel $, it outputs a common reference string $ crs $ and a trapdoor $ \tau $ for $ \rel $.
\item $\NIZK_\rel.\Prove(x; \omega) \mapsto \pi$ creates a proof $\pi$ for  $(x; \omega) \in \rel$. 
\item $\NIZK_\rel.\Verify(x; \pi)$ returns either true of false.
\item $ \NIZK_\rel.\Simulate(\tau,x) \mapsto \pi$ returns a proof $ \pi $.
\end{itemize}	
We note that $ \Prove $ and $ \Verify $ algorithms have $ crs $ and $ \rel $ as an input even though we do not explicitly specify them. We implicitly specify them by naming $ \NIZK_{\rel} $. $ \NIZK $ satisfies the following:

%We always describe circuits as languages $\lang$ and write $\NIZK_\lang$ for two reasons: 
%All SNARK circuits have many logic wires in \rel other than
%the public input wires $x$ and the secret input witness wires $\omega$.
%An existential quantifiers $\exists$ more clearly distinguishs
%public inputs $x$ from secret input witnesses $\omega$ than tuple position.
%We also benefited from language in the preceding informal exposition,
% which did not always require specifying $\omega$.

\begin{definition}\label{def:nizk_completeness}
We say $\NIZK_\rel$ is {\em complete} if $\Verify(x, \Prove(x; \omega))$ succeeds for all $(x; \omega) \in \rel$.  % with high probability
\end{definition}

\def\advV{\ensuremath{V^*}\xspace} % Why not use \adv here?

\begin{definition}\label{def:nizk_zero_knowledge}
We say $\NIZK$ is {\em perfect zero-knowledge} for all $ \secparam \in \mathbb{N} $, for all $ (\rel,z) \leftarrow \mathfrak{R} $, $ (x,\omega) \in \rel $ and for all adversaries $ \adv $, if the following holds:
\begin{align*}
	\Pr[& \adv(\crs, z, \pi,\rel) = 1 \ | \ (\crs, \tau) \leftarrow \NIZK.\Setup (\rel),   \pi \leftarrow \NIZK_\rel.\Prove (x;\omega)]\\
	=\Pr[& \adv(\crs, z, \pi,\rel) = 1 \ | \ (\crs, \tau) \leftarrow \NIZK.\Setup (\rel),   \pi \leftarrow \NIZK_\rel.\Simulate (\tau,x)]
\end{align*}

\end{definition}

\def\advP{\ensuremath{P^*}\xspace} % Why not use \adv here?

\begin{definition}\label{def:nizk_knowledge_sound}
We say $\NIZK$ is {computational knowledge sound} for all $ (\rel,z)  $ outputted by $ \mathfrak{R} $ if
for any non-uniform PPT adversary \adv there exists a PPT extractor $ \Extract $
such that
\begin{align}
	\Pr[\NIZK_\rel.\Verify(x,\pi) = 1 \wedge  (x;\omega) \notin \rel | (crs,\tau) \leftarrow \NIZK.\Setup(\rel),\nonumber\\ ((x,\pi);\omega)\leftarrow (\adv||\Extract)(\rel,z,crs)] =  \negl(\secparam) \nonumber
\end{align}
	
where $ (o_A;o_B)\leftarrow A||B(input) $ denote the algorithms that run on the same input and $ B $ has access the random coins of $ A $.
\end{definition}

%Our zero-knowledge continuations in \S\ref{sec:rvrf_cont} demand
%rerandomizing existing zkSNARKs, which only Groth16 supports \cite{Groth16}.
%We therefore introduce some details of Groth16 \cite{Groth16} there,
%when we tamper with Groth16's SRS and $\mathtt{Setup}$ to create zero-knowledge continuations. 
% TODO: Do we describe Groth16 \cite{Groth16} enough?

% In this, we exploit several arguments given by \cite{RandomizationGroth16},
% but for now we recall that \cite{RandomizationGroth16} proves that Groth16
% satisfies: % white-box weak simulation-extractablity .
%
% \begin{definition}\label{def:nizk_weak_simulation_extractable}
% We say $\NIZK_\rel$ is {\em white-box weak simulation-extractable} if
% for any non-uniform PPT adversary \advP with oracle access to \Simulate
% who outputs a statement $x \in \lang$ and proof $\pi$,
% there exists a PPT extractor algorithm $\Extract$ that white-box observes $\advP$ and
% if \advP never queried $x$ and $\Verify(x,\pi)$ holds
% then $\Extract$ returns an $\omega$ for which $(x; \omega) \in \rel$
% (see \cite[Def. 7, \S 2.3, pap. 29]{RandomizationGroth16}).
% \end{definition}

% TODO: AGM and Groth16 here?


%\subsection{Commitment Scheme}



%\subsection{Universal Composability (UC) Model}
%\label{subsec:uc_background}
%We define the security of ring VRFs in the UC model \cite{canetti1,canetti2}. In a nutshell, Canetti \cite{canetti1,canetti2} defines the UC model as follows:
%
%A protocol $ \phi $ in the UC model is an execution between distributed interactive Turing machines (ITM). Each ITM has a storage to collect the incoming messages from other ITMs, adversary \adv or the environment $ \env $. $ \env $ is an entity to represent the external world outside of the protocol execution.  The environment $ \env $ initiates ITM instances (ITIs) and the adversary \adv with arbitrary inputs and then terminates them to collect the outputs.
%% An ITM that is initiated by $ \env $ is called ITM instance (ITI). 
%We identify an ITI with its session identity $ \sid $ and its ITM's identifier $ \pid $. In this paper, when we call an entity as a party in the UC model we mean an ITI with the identifier $ (\sid, \pid) $.
%
%We define the ideal world where there exists an ideal functionality $ \mathcal{F} $ and the real world where a protocol $ \phi $ is run as follows:
%
%\paragraph{Real world:} $ \env $ initiates ITMs and \adv to run the protocol instance with some input $ z \in \{0,1\}^* $  and a security parameter $ \secparam $. After $ \env $ terminates the protocol instance, we denote the output of the real world by the random variable $ \mathsf{EXEC}(\secparam, z)_{\phi, \adv, \env} \in \{0,1\} $. Let $ \mathsf{EXEC}_{\phi, \adv, \env} $ denote the ensemble $ \{\mathsf{EXEC}(\secparam, z)_{\phi, \adv, \env} \}_{z \in \{0,1\}^*} $.
%
%\paragraph{Ideal world:} $ \env $ initiates ITMs and a simulator $ \simulator $ to contact with the ideal functionality $ \mathcal{F} $ with some input $ z \in \{0,1\}^* $  and a security parameter $ \secparam $. $ \mathcal{F} $ is trusted meaning that it cannot be corrupted.
%$ \simulator $ forwards all messages forwarded by $ \env $ to $ \mathcal{F} $. The output of execution with $ \mathcal{F} $ is denoted by a random variable $ \mathsf{EXEC}(\secparam, z)_{\mathcal{F},\simulator, \env} \in \{0,1\}$.  Let $ \mathsf{EXEC}_{\mathcal{F},\simulator, \env} $ denote the ensemble $ \{\mathsf{EXEC}(\secparam, z)_{\mathcal{F}, \simulator, \env} \}_{z \in \{0,1\}^*} $.
%
%%TODO: \secparam should likely be implicit, especially since it appears in both worlds.
%
%\begin{definition}[UC-Security of $ \phi $] \label{def:uc}
%	Given a real world protocol $ \phi $ and an ideal functionality $ \mathcal{F} $ for the protocol $ \phi $, we call that $ \phi $ is UC-secure i.e., $ \phi $ UC-realizes $ \mathcal{F} $, if for all PPT adversaries \adv,  there exists a simulator $ \simulator  $ such that for any environment $ \env $,
%	$\mathsf{EXEC}_{\phi, \adv, \env}$ is indistinguishable from $\mathsf{EXEC}_{\mathcal{F},\simulator, \env}$.
%\end{definition}



%\begin{definition}[UC-Security of $ \phi $ in the hybrid world]
%	Given a real world protocol $ \phi $ which runs some (polynomially many) functionalities $ \{\mathcal{F}_1, \mathcal{F}_2, \ldots, \mathcal{F}_k\} $ in the ideal world and an ideal functionality $ \mathcal{F} $ for the protocol $ \phi $, we call that $ \phi $ is UC-secure in the hybrid model $ \{\mathcal{F}_1, \mathcal{F}_2, \ldots, \mathcal{F}_k\} $ if $ \phi $ UC-realizes $ \mathcal{F} $ if for all PPT adversaries \adv, there exists a simulator $ \simulator  $ such that for any environment $ \env $,
%	$\mathsf{EXEC}_{\phi, \adv, \env}$ is indistinguishable from $\mathsf{EXEC}_{\mathcal{F},\simulator, \env}$.
%\end{definition}

% REMARKS:  Removed excessive notation $\approx$.

