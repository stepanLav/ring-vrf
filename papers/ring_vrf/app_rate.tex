\section{Application: Rate limiting}
\label{sec:app_rate_limits}

We showed in \S\ref{sec:app_identity} how ring VRFs give users only
one unique identity for each input \msg.  
We explained in \S\ref{subsec:moderation} that choosing \msg to be
the concatenation of a base domain and a date gives users a stream of changing identities.

We next discuss giving users exactly $n > 1$ ring VRF outputs aka
``identities'' per date, as opposed to the unique identity 


% \subsection{Implementation}

As a trivial implementation, we could include a counter $k = 1 \ldots n$
in \msg, so $\msg = \mathtt{domain} \doubleplus \mathtt{date} \doubleplus k$.


\subsection{Avoiding linkage}

Our trivial implementation leaks information about ring VRF outputs'
 ownership by revealing $k$:
%
An adversary Eve observes two ring VRF signatures with the same
$\mathtt{domain}$ and $\mathtt{date}$ so
$\msg_i = \mathtt{domain} \doubleplus \mathtt{date} \doubleplus k_i$
for $i=1,2$, but with different outputs $\Out_1$ and $\Out_2$.
If $k_1 \ne k_2$ then Eve learns nothing, but if $k_1 = k_2$ then
 Eve learns that $sk_1 \ne \sk_2$, maybe representing different users. 

We do not necessarily always care if Eve learns this much information,
but scenarios exist in which one cares.  We therefore briefly describe
several mitigations:

If $n$ remains fixed forever, then we could simply let all users
register $n$ ring VRF public keys in \ctx.
If $n$ fluctuates under an upper bound $N$, then we could create $N$
rings $\ctx_i$ for $i = 1 \ldots N$, and
 then blind \comring in \pifast similarly to \S\ref{sec:ring_hiding}.

Although simple, these two approaches require users construct $n$ or $N$
different $\pipk$ proofs every time the ring \ctx updates.

Instead of proving ring membership of one public key, $\pipk$ could
prove ring membership of a Merkle commitment to multiple keys, so
users have $\pisk^1,\ldots,\pisk^N$ for each of their multiple keys.

As a more flexible approach,
we could compute the hash-to-curve $\In := H_{\grE}(\msg)$ inside an
unamortized SNARK $\pi_{\mathtt{in}}$ and reveal only a Pedersen-like commitment
to $\In + \openpk^{-1} \genB$.  We then adjust \PedVRF to yield
a proof-of-knowledge of $\Out/\In$ subject to soundness of this
SNARK $\pi_{\mathtt{in}}$.

TOTO: Explain better!!

In all cases, we incur costs by hiding part of the input \msg, so
deployment should seriously consider if leaking $k$ suffices.


\subsection{Ration cards}

As a species, we expect $+3^{\circ}$ C or more likely $+4^{\circ}$ C
over the pre-industrial climate by 2100 \cite{IPCC2022}, which shall
reduce the Earth's carrying capasity below one billion people \cite{carrying_capasity}.
In the shorter term, we expect shortages of resources, energy, goods,
water, and food beginning during the next several decades, due to
climate change, ecosystem damage or collapse, and resource exhaustion
ala peak oil.  Invariably, nations manage shortages through rationing,
like during WWI, WWII, and the oil shocks.  

Ring VRFs support anonymous rationing:
Instead of treating ring VRF outputs like identities,
we treat them like nullifiers which could each be spent exactly once.

\def\expiry{e}
We fix a set $U$ of limited resource types, and dynamically define
an expiry date $\expiry_{u,d_0}$ and an availability $n_{u,d_0}$, 
both dependent upon the resource $u \in U$ and current date $d_0$.
We typically want a randomness beacon $r_d$ too, which prevents
anyone learning $r_d$ much before date $d$. 
% Among other usages, this reduces damages from key compromises.
As ring VRF inputs, we choose
 $\msg = u \doubleplus r_d \doubleplus d \doubleplus k$
where $u \in U$ denotes a limited resource,
 $d$ denotes an non-expired date meaning $\expiry_{u,d_0} < d \le d_0$,
 and $1 \le k \le n_{u,d_0}$.
In this way, our rationing system controls both daily consumption
via $n_{u,d_0}$ and time shifted demand via expiry time $\expiry_{u,d_0}$.

Importantly, our rationing system retains ring VRF outputs as nullifiers,
filed under their associated date $d$ and resource $u$, so nullifiers
expire once $d \le \expiry_{u,d_0}$ which permits purging old data rapidly.

We remark that fully transferable assets could have constrained lifetimes
too, which similarly eases nullifier management when implements using
blind signatures, zcash sapling, etc.  Yet, all these tokens require
an explicit issuance stage, while ring VRFs self-issue.

Among the political hurdles to rationing, we know certificates have
a considerable forgery problem, as witnessed by the long history of
fraudulent covid and TLS certificates.  It follows citizens would
justifiably protest to ration carts that operate by simple certificates.
Ring VRFs avoid this political unrest by proving membership in a public list.


\subsection{Multi-constraint rationing}

We could proving multiple \ThinVRF outputs with one signatures
in \S\ref{subsec:vrf_thin}.  We needed \PedVRF to isolate the blinding 
factor when using a Pedersen commitment instead of a public key, but
exactly the same technique works for proving multiple \PedVRF outputs
in one ring VRF signature.

We could therefore impose simultanious rationing constraints for multiple
resources $u_1,\ldots,u_k$ by producing one ring VRF signature in which
\PedVRF proves correctness of pre-ouptuts for multiple messages 
 $\msg_j = u_j \doubleplus r_d \doubleplus d \doubleplus k$ for $j=1 \ldots k$.

As an example, purchasing some prepared food product could require espending
rations for multiple base food sources, like making a cake from wheat, butter,
eggs, and sugar.  


\subsection{Decommodification}

There exist many reasons to decommodify important services,
like energy, water, or internet,
 beyond rationing real physical shortage.
Ring VRFs fit these cases using similar \msg formulations.

As an example, a municipal ISP allocates some limited bandwidth capacity
among all residents.  It allocates bandwidth fairly by verifying ring VRFs
signatures on hourly \msg and then tracking nullifiers until expiry.

Aside from essential government services, commercial service providers
typically offers some free service tier, usually because doing so
familiarizes users with their intimidating technical product.

Some free and paid tier examples include DuoLingo's heats on mobile, 
continuous integration testing services, and many dating sites.

A priori, rate limiting cases benefit from unlinkability among individual
usages, not merely at some site boundary like moderation requires.
We thus use each ring VRF output only once, which prevents our cashing
trick of \S\ref{sec:reduced_pairings} from reducing verifier pairings.

Although rationing sounds valuable enough, we foresee services like ISP,
VPNs, or mixnets having many low value transactions.
In such cases, ring VRFs could authorize issuing a limited number of
fast simple single-use blind issued credentials, like blind signatures
ala GNU Taler \cite{taler} or PrivacyPass OPRF tokens \cite{PrivacyPass},
 which both solve the leakage of $k$ above too.

In principle, commercial service providers could sell the same tokens,
which avoids leaking whether the user uses the free or commercial tier.


\subsection{Delegation}

We sometimes want to delegate spending ring VRF outputs, without
creating a fully transferable asset.  In particular, parents might
delegate limited internet or streaming service access to a child,
but without making the token full transferable.

Among blind issued credential many support this, both
GNU Taler \cite{taler} and PrivacyPass \cite{PrivacyPass} could
be redeemed by a delegatee family member who trusts the original
delegator not to double spend, but transactions with untrusted spenders
risks double spending.  

Ring VRFs usage typically demands spender authenticate the specific
spending operations inside the the associated data \aux, but adjusting
\aux requires knowing \sk, perhaps unacceptable to the delegator.

We could however achieve delegation by treating the ring VRF like a
certificate that authenticates another public key held by the delegatee.
GNU Taler achieves delegation and other features like this.

We could similarly treat the ring VRF like an adaptor certificate aka
implicit certificate.  In other words, the delegatee learns the full
ring VRF signatures, but then the delegatee hides $s_1$ from downstream
recipients, and instead merely prove knowledge of $s_1$, usually via
a key exchange or another Schnorr signature.

